\documentclass[UTF8,a4paper]{ctexart}

\usepackage{geometry}
\usepackage{setspace}
\usepackage{fancyhdr}
\usepackage{titlesec}
\usepackage{graphicx}
\usepackage{booktabs}

% Page layout
\geometry{left=2.5cm,right=2.5cm,top=2cm,bottom=2cm}
\setlength{\parindent}{2em}
\setstretch{1.5}

% Fonts
\setCJKmainfont{SimSun}
\ctexset{
  section = {format=\heiti\zihao{2}},
  subsection = {format=\heiti\zihao{3}},
  subsubsection = {format=\heiti\zihao{4}}
}

% Footer page number (centered, small-five)
\pagestyle{fancy}
\fancyhf{}
\fancyfoot[C]{\songti\zihao{5}\thepage}

\begin{document}
\songti\zihao{-4}

\begin{titlepage}
\thispagestyle{empty}

\begin{center}
  \vspace*{2cm}
  {\heiti\zihao{3} 西北工业大学《移动应用开发》}\\[1.5em]
  {\heiti\zihao{2} 大作业报告}
\end{center}

\vspace{5cm}

\noindent
题目:

\vspace{2cm}

\noindent
\begin{tabular}{@{}ll}
学生姓名:& XXX\\
学号:& XXXXXXXX\\
班级:& XX班\\
指导教师:& XXX\\
提交日期:& XXXX年XX月XX日
\end{tabular}
\end{titlepage}

\section{选题依据}
本项目为“采摘工智慧管理系统/采摘通(Picking-Pass)”,面向农业生产基地与采摘工人的用工场景,解决传统园区“招工难、考勤乱、结算慢”的痛点。系统覆盖岗位发布、工人报名、扫码核销到薪资自动结算的全流程,既贴近真实生产管理需求,也具备明确的数据库应用背景。项目强调数据驱动与流程闭环:通过二维码签到与工时记录打通用工统计,基于历史数据与岗位画像进行智能推荐;对敏感信息采用“密文存储+Hash索引”方案保护隐私,同时保证检索性能;数据库层面使用触发器保障业务一致性。该项目与《数据库系统》课程的范式设计、事务一致性、索引与安全等核心知识高度相关,具有较强的工程实践和创新价值。

\section{设计思路}
项目采用前后端分离的 B/S 架构,包含 Web 管理端、移动管理端以及微信小程序端,形成多角色、多端协同的业务体系。总体目标是实现“用工信息全生命周期管理”:从基地入驻、岗位发布、工人报名、现场签到、工时记录到工资结算与发放,形成可追溯的数据闭环。

技术选型方面:后端使用 NestJS + TypeORM + MySQL 8.0,提供模块化业务能力与 API 文档(Swagger);前端使用 React + Vite + TypeScript + TailwindCSS;小程序端负责报名、签到与信息查询。权限体系采用 RBAC 角色模型(超级管理员、区域管理员、基地管理员、现场管理员、工人),并配合操作日志实现审计追溯。数据库遵循 3NF 设计,核心表包括用户、基地、岗位、考勤、工资及合作申请等,配合触发器与单价快照机制,确保历史数据一致性。系统模块划分清晰,包括用户信息管理、基地管理、岗位申请、考勤签到、薪资结算、智能推荐与公共服务等。

\section{实现过程}
1) \textbf{环境与数据库初始化}:配置 Node.js 与 MySQL 环境,创建数据库并导入初始化 SQL;设置后端环境变量,启动 NestJS 服务与前端开发服务器。\\
2) \textbf{核心业务模块实现}:完善用户信息管理(含紧急联系人与审核流程),实现三种录入方式(手动、OCR、分步引导);加入 AES-256 加密存储与 Hash 索引检索;实现基地入驻审核、岗位发布、岗位申请与双向选择机制;考勤模块支持报名确认短信、动态二维码签到及离线同步;工资结算支持固定/计件/时薪三种策略,生成工资确认单并记录支付凭证。\\
3) \textbf{系统增强与扩展}:实现操作日志与自动备份机制;预留地图服务与评价系统接口;补充智能推荐模块,基于区域、历史偏好与活跃岗位进行加权推荐。\\
4) \textbf{测试与联调}:依据测试指南与联调清单完成多角色测试,包括注册登录、基地审核、岗位发布/申请、签到与结算等;部分展示型页面采用静态数据或 mock 数据,核心链路均可用。

关键难点及解决方法:其一是敏感信息加密与高效查询的平衡,采用“密文存储+Hash索引”解决;其二是二维码签到与离线同步的一致性问题,通过离线缓存与批量同步接口处理;其三是工资计算策略多样化,使用策略模式并保留单价快照防篡改。

\section{对课程的建议}
建议在课程中增加真实业务案例的数据建模与范式设计练习,强化“需求—ER 模型—表结构—索引/触发器”的完整训练;同时适当增加接口与前后端联调的实践环节,让数据库设计与业务实现结合更紧密;可提供更细化的测试清单模板与评分标准,帮助同学更系统地完成验证与总结。

\section{附录}
1. 核心代码片段(关键功能实现代码,标注说明):\\
\quad 后端模块:\texttt{app/backend/src/modules/user/}、\texttt{attendance/}、\texttt{salary/}、\texttt{base/}、\texttt{recommendation/}。\\
\quad 触发器与数据库:\texttt{app/backend/picking\_pass\_db\_final.sql}。\\
2. 小程序运行截图/演示视频链接(含运行环境、核心功能演示画面):待补充。\\
3. 参考文献(开发过程中参考的官方文档、技术博客等):\\
\quad NestJS 官方文档、TypeORM 官方文档、MySQL 8.0 文档、微信小程序开发文档。

\vspace{1em}

格式要求:\\
- 字体与字号:标题(一级标题:黑体二号,二级标题:黑体三号,三级标题:黑体四号);正文(宋体小四,行间距:1.5倍);页码(页脚居中,宋体小五)。\\
- 排版要求:正文段落首行缩进2字符,左右页边距均为2.5厘米,上下页边距均为2厘米;图表需标注编号与名称(如“图1 智能体整体架构图”“表1 核心功能测试用例表”),图表名称居中位于图表下方。\\
- 原创性要求:报告内容需为学生独立撰写,严禁抄袭;核心代码需为自主开发,若参考他人代码需注明出处。

\end{document}
