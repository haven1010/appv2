\documentclass[UTF8,a4paper,12pt]{article}
\usepackage{ctex}         % 中文支持
\usepackage{geometry}     % 页面布局
\usepackage{graphicx}     % 插入图片
\usepackage{float}        % 图片位置固定
\usepackage{listings}     % 代码块高亮
\usepackage{xcolor}       % 颜色支持
\usepackage{amsmath}      % 数学公式
\usepackage{amssymb}      % 数学符号
\usepackage{booktabs}     % 表格线美化
\usepackage{hyperref}     % 超链接
\usepackage{enumitem}     % 列表格式
\usepackage{setspace}
\usepackage[table]{xcolor} % 支持表格行底色
\usepackage{caption}       % 优化标题
\usepackage{longtable}     % 支持长表格(如果需要)
\usepackage{array}         % 优化表格列宽

% 字体设置:避免等宽字体缺失框线字符
\setmonofont{Consolas}
\setCJKmonofont{NSimSun}

% 页面布局设置
\geometry{left=2.5cm,right=2.5cm,top=2.5cm,bottom=2.5cm}

\lstset{
    language=Java, % 适配JavaScript/SQL
    basicstyle=\ttfamily\small,
    tabsize=4,
    breaklines=true,
    columns=flexible,
    frame=single,
    framerule=0.5pt,
    numbers=left,
    numberstyle=\tiny\color{gray},
    keywordstyle=\color{blue}\bfseries,
    commentstyle=\color{gray}\itshape,
    stringstyle=\color{red!80!black},
    backgroundcolor=\color{gray!5},
    escapeinside=``,
}

\hypersetup{
    colorlinks=true,
    linkcolor=black,
    anchorcolor=black,
    citecolor=black
}

\begin{document}

% 封面页
\begin{titlepage}
    \centering
    
    % --- 第一部分:学校/单位信息 ---
    \vspace*{1cm}
    {\Large \scshape 西北工业大学计算机学院} 
    {\Large \scshape 《数据库系统》课程大作业报告} \\

    
    \vspace{3cm}
    
    % --- 第二部分:主标题 ---
    \begin{spacing}{1.5} % 需要在导言区引用 \usepackage{setspace}
        \textbf{\Huge 采摘工智慧管理系统(采摘通)设计与开发} \\
        \vspace{0.5cm}
        \textbf{\Large CaiZhaiTong (Picking-Pass) Management System}
    \end{spacing}
    
    \vspace{4cm}
    
    % --- 第三部分:团队信息(表格形式更整齐) ---
    \begin{minipage}{0.6\textwidth}
        \large
        \begin{tabular}{ll}
            \textbf{专\quad 业:} & 计算机科学与技术 \\
            \textbf{学\quad 院:} & 计算机学院 \\
            \textbf{成\quad 员:} & 赵张阳(2023302782) \\
            \textbf{}            & 张睿(2023300395) \\
            \textbf{指导教师:}   & 李宁 \\
        \end{tabular}
    \end{minipage}

    \vfill % 自动填充余下空间
    
    % --- 第四部分:日期 ---
    {\large 2025年12月}
    
    \thispagestyle{empty}
\end{titlepage}

\newpage
\tableofcontents
\clearpage

\section{选题背景与意义}

\subsection{行业背景与现状痛点}
随着现代农业规模化与集约化的发展,农业生产对灵活用工的需求日益增长。然而,传统的农业劳动力管理模式仍停留在“粗放式”阶段,存在以下显著痛点:

\begin{itemize}
    \item \textbf{信息不对称}:农场(基地)的用工需求与工人的求职意向缺乏高效的对接渠道,导致“招工难”与“找活难”并存。
    \item \textbf{考勤与结算混乱}:农业用工具有极强的流动性和季节性,且薪资计算方式复杂(涵盖固定薪资、计时、计件等多种模式)。依靠纸质或口头约定的传统记账方式,极易引发劳资纠纷。
    \item \textbf{管理层级断层}:从区域管理员到基地负责人,再到现场管理员,缺乏统一的数据化管理平台,导致监管数据滞后,无法实时掌握各基地的用工成本与效率。
\end{itemize}

\subsection{业务逻辑的复杂性挑战}
本系统旨在解决农业场景下复杂的“人-地-事-钱”关系。根据实体关系分析(ER分析),系统需处理高度关联的业务闭环,这对数据库的逻辑设计提出了极高要求:

\begin{itemize}
    \item \textbf{多维度的实体关联}:系统核心围绕 \textbf{用户表 (\texttt{sys\_user})}、\textbf{基地表 (\texttt{base\_info})}、\textbf{岗位表 (\texttt{recruitment\_job})} 三大核心实体展开。其中,岗位与基地呈“一对多”的强依赖关系,而用户与岗位之间则通过 \textbf{报名记录表 (\texttt{daily\_signup})} 建立时空约束下的“多对多”关联。
    
    \item \textbf{全流程的状态流转}:从工人的“报名 (Status=0)”到“现场签到 (Status=1)”,再到“薪资结算”,业务链路长且环环相扣。任一环节的状态变更(如缺勤或取消)都必须严格触发后续流程的联动,这需要依赖数据库的事务(Transaction)或状态机逻辑来保证数据一致性。
    
    \item \textbf{灵活的薪资体系}:系统必须支持基于 \textbf{薪资表 (\texttt{labor\_salary})} 实体的复杂结算逻辑。既要处理基于 \texttt{work\_duration} 的工时计算,又要支持基于 \texttt{piece\_count} 的计件统计,同时还需保证报名记录与薪资记录严格的“一对一”外键约束关系,以确保财务数据的准确性。
\end{itemize}

\subsection{技术架构选型与实施必要性}
面对上述复杂的业务需求与数据关系,传统的单体应用或简单的 CRUD 系统已无法满足需求。本课程设计采用 \textbf{NestJS + TypeORM + MySQL 8.0} 的后端技术栈,并结合前端管理端(React 19 + Vite 6)与微信小程序,形成多端协同的完整业务闭环。在数据库设计层面具有重要的实践意义:

\begin{itemize}
    \item \textbf{规范化的数据库建模}:NestJS 的模块化架构配合 TypeORM 实体映射,将业务概念落地为实体表模型,支持 \texttt{@OneToMany}(基地-岗位)、\texttt{@ManyToOne}(岗位-基地)、\texttt{@OneToOne}(报名-薪资)、\texttt{@ManyToOne}(申请/合作/支付)等关系映射,保持模型清晰一致。
    
    \item \textbf{一致性与状态流转控制}:报名、签到、工资结算、支付确认等流程在服务层通过事务与状态机约束实现联动,结合数据库外键与唯一性约束(如 \texttt{daily\_signup(user\_id, work\_date)})保证业务闭环的正确性。
    
    \item \textbf{安全与隐私保护}:系统在 \texttt{sys\_user}、\texttt{base\_info} 等实体中对身份证号、手机号、紧急联系人、营业执照等敏感字段进行 AES-256 加密存储,并对手机号/身份证/紧急联系人手机号建立哈希索引(\texttt{\_hash}),兼顾安全性与检索性能。
\end{itemize}


\section{需求分析}

\subsection{数据需求分析}
根据数据库模式设计,系统核心数据实体及其关键属性如下:

\begin{itemize}
    \item \textbf{用户信息数据 (sys\_user)}:
    需存储用户的全局唯一标识 (\texttt{uid})、真实姓名 (\texttt{name}) 以及加密存储的敏感信息(身份证号 \texttt{id\_card\_enc}、手机号 \texttt{phone\_enc}、紧急联系人 \texttt{emergency\_contact\_enc}、紧急联系人电话 \texttt{emergency\_phone\_enc})。系统需支持基于哈希值 (\texttt{phone\_hash}, \texttt{id\_card\_hash}, \texttt{emergency\_phone\_hash}) 的快速检索,并区分五种角色权限:超级管理员、区域管理员、基地管理员、现场管理员及普通工人 (\texttt{worker}),同时支持字段 \texttt{assigned\_base\_id} 与 \texttt{info\_audit\_status} 进行现场管理与信息审核控制。

    \item \textbf{基地信息数据 (base\_info)}:
    需记录基地的基本属性(名称、地址、区域代码 \texttt{region\_code})及经营类别(1:果类, 2:菜类, 3:其他),同时支持基地简介与多媒体描述 (\texttt{description})。基地需管理审核状态 (\texttt{audit\_status}),确保持证经营,并支持加密存储营业执照 (\texttt{license\_enc}) 与联系方式 (\texttt{contact\_enc})。

    \item \textbf{岗位与招聘数据 (recruitment\_job)}:
    需定义具体的招聘岗位信息,包括岗位名称、所属基地、招聘人数、工作内容、工作周期与工作时间。薪资体系 (\texttt{pay\_type}) 支持固定薪资 (\texttt{salary\_amount})、时薪 (\texttt{hourly\_rate}) 与计件单价 (\texttt{unit\_price}) 三种模式,并包含目标计件数 (\texttt{targetCount})、年龄限制、经验要求、体力要求、福利保障、住宿/餐食/交通补贴等属性;同时支持岗位有效期、自动续期与多媒体展示字段。

    \item \textbf{报名与考勤数据 (daily\_signup)}:
    需记录工人每日的报名状态流转:从“已报名(0)”到“已签到(1)”或“缺勤/取消”。系统支持代理报名 (\texttt{is\_proxy} 与 \texttt{proxy\_user\_id}) 及离线同步 (\texttt{is\_offline\_sync}) 标记,且需通过唯一约束 (\texttt{user\_id} + \texttt{work\_date}) 防止重复报名。

    \item \textbf{薪资结算数据 (labor\_salary)}:
    需记录与报名记录一一对应的结算详情。根据岗位类型存储工时 (\texttt{work\_duration}) 或计件量 (\texttt{piece\_count}),并保留计算时的单价快照 (\texttt{unit\_price\_snapshot}) 以计算总金额。此外,需追踪支付状态与支付方式,并记录支付凭证 (\texttt{proof\_img\_url}) 与工人签字 (\texttt{worker\_sign\_url})。
    
    \item \textbf{岗位申请与合作数据 (job\_application / base\_cooperation)}:
    系统支持工人对岗位的申请与基地的双向合作申请,需记录申请状态、审核人、审核时间、备注/拒绝原因等字段,形成“申请—审核—反馈”的闭环链路。
    
    \item \textbf{工资发放数据 (salary\_payment)}:
    需记录工资发放方式(现金/转账)、确认签字、发放凭证、发放人及发放时间,用于对账与审计。
    
    \item \textbf{系统审计数据 (operation\_log)}:
    对关键业务操作进行审计记录,包括操作类型、资源类型、操作人、操作前后数据、IP 与 User-Agent。
    
    \item \textbf{基地评价数据 (base\_rating)}:
    作为预留扩展表,支持用户对基地的评分与评价标签,完善口碑体系。
\end{itemize}

\subsection{功能需求分析}
基于数据实体,系统需实现以下核心业务功能:

\begin{enumerate}
    \item \textbf{多角色认证与权限管理}:
    支持基于手机号 + 身份证后六位的登录认证,系统按 \texttt{role\_key} 进行 RBAC 权限控制。超级管理员拥有全局视野,区域管理员管理特定 \texttt{region\_code} 的基地,基地管理员负责基地信息与岗位发布,现场管理员负责签到与工量录入,工人仅访问个人信息与岗位报名。

    \item \textbf{用户注册与信息审核}:
    支持手动注册、OCR 注册与补全注册三种入口;用户信息更新后进入审核状态 (\texttt{info\_audit\_status}),保证人员信息的真实性与可追溯性。

    \item \textbf{基地与岗位全生命周期管理}:
    管理员可创建并维护基地信息,提交审核。审核通过后,可发布招聘岗位并配置工作周期、年龄要求、薪资模式与福利保障,同时支持岗位有效期、续期与上下架控制。

    \item \textbf{报名、签到与代报名流程}:
    工人可报名有效岗位,系统自动校验是否重复报名;现场管理员扫码核销或手动确认签到,支持代报名与离线同步,保证弱网场景下数据可回传。

    \item \textbf{薪资结算与发放闭环}:
    管理端录入工时/计件量后触发工资自动计算并生成结算记录;工资发放模块记录确认签字与发放凭证,并支持现金/转账等发放方式。

    \item \textbf{岗位申请与合作机制}:
    工人可对岗位发起申请,基地管理员审核反馈;区域/超级管理员可对基地发起合作申请,形成双向筛选与合作闭环。

    \item \textbf{操作日志与统计报表}:
    对关键操作进行审计记录(\texttt{operation\_log}),并提供管理端仪表盘与报表统计功能,支持按基地、区域、时间维度查询。
\end{enumerate}

\subsection{非功能需求分析}
\begin{itemize}
    \item \textbf{数据安全与隐私}:
    敏感字段(手机号、身份证、紧急联系人、营业执照)必须在应用层加密后存入数据库,数据库中仅存储密文及用于检索的哈希值,严禁明文存储。
    
    \item \textbf{业务一致性保障}:
    利用数据库事务及外键约束,确保“无报名不签到,无签到不结算”。同时通过服务层状态机约束报名/薪资/支付流程,防止越级流转或重复发放。
    
    \item \textbf{高并发与离线支持}:
    针对早高峰签到场景,系统需支持快速响应(API 响应 $<1$s)。\texttt{is\_offline\_sync} 字段支持弱网环境下的数据暂存与网络恢复后的自动同步功能;系统同时支持短信通知与二维码核验的可靠性保障。
    
    \item \textbf{可运维性}:
    后端提供操作日志与定时备份机制(日志追踪 + 定时备份),便于问题定位与数据恢复。
\end{itemize}

% ==========================================
% 3. 概念结构设计
% ==========================================
\section{概念结构设计}

\subsection{实体与属性定义}
根据需求分析,系统主要包含以下实体及其核心属性:

\begin{description}
    \item[用户 (User)] 属性:用户ID、UID、姓名、加密身份证号、加密手机号、紧急联系人、角色、哈希索引值、信息审核状态。
    \item[基地 (Base)] 属性:基地ID、基地名称、营业执照、联系方式、区域代码、经营类别、审核状态、详细地址、简介描述。
    \item[岗位 (Job)] 属性:岗位ID、岗位标题、薪资类型(固定/时薪/计件)、单价、工作周期、福利待遇、有效期、多媒体展示。
    \item[报名 (Signup)] 属性:报名ID、工作日期、状态(报名/签到/缺勤/取消)、签到时间、是否代报名、离线同步标记。
    \item[薪资 (Salary)] 属性:薪资ID、工时/件数、单价快照、总金额、支付状态、支付方式、凭证与签字。
    \item[申请 (JobApplication)] 属性:申请ID、申请状态、审核人、审核时间、备注/拒绝原因。
    \item[合作申请 (BaseCooperation)] 属性:合作ID、申请人、需求描述、审核状态。
    \item[工资发放 (SalaryPayment)] 属性:发放ID、发放方式、签字、凭证、发放时间。
    \item[操作日志 (OperationLog)] 属性:操作类型、资源类型、操作人、前后数据、IP。
    \item[基地评价 (BaseRating)] 属性:评分、评价内容、标签(预留)。
\end{description}

\subsection{实体间联系 (ER图描述)}
各实体之间的联系如下:

\begin{itemize}
    \item \textbf{用户与报名} (1:N):一个用户可以在不同日期进行多次报名,但同一天同一岗位只能报名一次。
    \item \textbf{基地与岗位} (1:N):一个基地可以发布多个招聘岗位,一个岗位仅属于一个基地。
    \item \textbf{岗位与报名} (1:N):一个岗位可以接收多个工人的报名。
    \item \textbf{报名与薪资} (1:1):一条有效的报名记录(已签到)对应且仅对应一条薪资结算记录。
    \item \textbf{用户与基地} (1:N):特定角色的用户(基地管理员)可以管理一个或多个基地(通过 \texttt{owner\_id} 关联)。
    \item \textbf{用户与岗位申请} (1:N):工人可对多个岗位发起申请,岗位申请可被基地管理员审核。
    \item \textbf{基地与合作申请} (1:N):区域/超级管理员可向基地发起合作申请,基地审核合作关系。
    \item \textbf{薪资与发放} (1:N):一条薪资结算记录可产生对应的工资发放记录(确认与发放过程)。
    \item \textbf{操作日志与业务实体} (N:1):操作日志记录与用户、基地、岗位、报名、薪资等资源关联。
\end{itemize}

\begin{figure}[H]
    \centering
    % 请在本地生成 ER 图图片后取消注释并替换文件名
    \includegraphics[width=0.9\textwidth]{er_diagram.png}
    \caption{系统 E-R 图}
    \label{fig:er_diagram}
\end{figure}

% ==========================================
% 4. 逻辑结构设计
% ==========================================
\section{逻辑结构设计}

\subsection{关系模式定义}
将 E-R 图转换为关系模型,主键用\textbf{加粗}表示,外键用\underline{下划线}表示。所有模式均满足第三范式 (3NF)。

\begin{enumerate}
    \item \textbf{系统用户} (\textbf{id}, uid, name, id\_card\_enc, phone\_enc, id\_card\_hash, phone\_hash, role\_key, assigned\_base\_id, emergency\_contact\_enc, emergency\_phone\_enc, emergency\_phone\_hash, info\_audit\_status, ...)
    \item \textbf{基地信息} (\textbf{id}, base\_name, license\_enc, contact\_enc, category, region\_code, address, description, audit\_status, \underline{owner\_id}, ...)
    \item \textbf{招聘岗位} (\textbf{id}, \underline{base\_id}, job\_title, pay\_type, unit\_price, salary\_amount, hourly\_rate, work\_cycle, status, valid\_until, ...)
    \item \textbf{每日报名} (\textbf{id}, \underline{user\_id}, \underline{base\_id}, \underline{job\_id}, work\_date, status, checkin\_time, is\_proxy, proxy\_user\_id, is\_offline\_sync, ...)
    \item \textbf{劳动薪资} (\textbf{id}, \underline{signup\_id}, work\_duration, piece\_count, unit\_price\_snapshot, total\_amount, payout\_type, status, proof\_img\_url, worker\_sign\_url, admin\_id, ...)
    \item \textbf{岗位申请} (\textbf{id}, \underline{user\_id}, \underline{job\_id}, \underline{base\_id}, status, note, reject\_reason, reviewed\_by, reviewed\_at, ...)
    \item \textbf{基地合作} (\textbf{id}, \underline{base\_id}, \underline{applicant\_id}, requirement, status, reject\_reason, reviewed\_by, reviewed\_at, ...)
    \item \textbf{工资发放} (\textbf{id}, \underline{salary\_id}, payment\_method, status, confirm\_signature\_url, payment\_voucher\_url, paid\_at, paid\_by, note, ...)
    \item \textbf{操作日志} (\textbf{id}, operation\_type, resource\_type, \underline{resource\_id}, \underline{user\_id}, before\_data, after\_data, ip\_address, user\_agent, ...)
    \item \textbf{基地评价} (\textbf{id}, \underline{base\_id}, \underline{user\_id}, rating, comment, tags, ...)
\end{enumerate}

\subsection{规范化分析}
\begin{itemize}
    \item \textbf{1NF}:所有字段均为原子值,如福利待遇虽为 JSON 格式,但在业务逻辑中视为整体存储,不影响原子性。
    \item \textbf{2NF}:所有表的主键均为单列自增 ID,不存在非主属性对主键的部分依赖。
    \item \textbf{3NF}:通过外键引用(如 \texttt{labor\_salary} 引用 \texttt{daily\_signup} 而非直接引用 \texttt{sys\_user}),消除了传递依赖。
\end{itemize}

% ==========================================
% 5. 安全性与完整性设计
% ==========================================
\section{安全性与完整性设计}

\subsection{数据库安全性设计}
\begin{enumerate}
    \item \textbf{敏感数据加密}:
    针对用户的身份证号、手机号、紧急联系人及基地资质图片,不在数据库中存储明文。应用层使用 AES-256 加密后存入 \texttt{\_enc} 字段,并为手机号/身份证/紧急联系人手机号生成 SHA-256 哈希值存入 \texttt{\_hash} 字段用于检索,确保数据泄露时不可逆。
    
    \item \textbf{防 SQL 注入}:
    后端采用 TypeORM 框架,通过参数化查询 (Parameterized Queries) 和对象关系映射机制,自动转义输入参数,从根本上杜绝 SQL 注入攻击。
    
    \item \textbf{访问控制}:
    利用 \texttt{role\_key} 字段实现 RBAC (Role-Based Access Control) 模型,结合 JWT 鉴权与权限守卫控制接口访问范围,同时将关键操作写入 \texttt{operation\_log} 以便审计追踪。
\end{enumerate}

\subsection{数据完整性设计}
\begin{itemize}
    \item \textbf{实体完整性}:所有表均设置 \texttt{id} 为主键 (Primary Key) 并开启 \texttt{AUTO\_INCREMENT}。
    \item \textbf{参照完整性}:通过外键约束 (Foreign Key) 确保数据一致性。例如,\texttt{daily\_signup} 表中的 \texttt{job\_id} 必须存在于 \texttt{recruitment\_job} 表中,且设置级联规则防止误删。
    \item \textbf{域完整性}:使用 \texttt{ENUM} 类型限制 \texttt{role\_key} 的取值范围;使用 \texttt{TINYINT} 配合 DEFAULT 值控制 \texttt{status} 状态机。
\end{itemize}

\subsection{业务联动与事务设计}
项目当前未启用数据库触发器,报名—签到—结算—发放流程由服务层事务与状态机逻辑保障一致性,避免误触发或跨服务联动导致的数据污染。典型联动规则包括:

\begin{itemize}
    \item 报名记录取消后,禁止生成薪资记录;已生成的工资记录需标记为待确认或取消。
    \item 薪资计算时锁定报名记录并写入单价快照,防止岗位单价调整影响历史账单。
    \item 工资发放记录需经工人签字确认后方可变更为“已发放”状态,并记录凭证与发放人。
\end{itemize}

% ==========================================
% 6. 物理结构设计
% ==========================================
\section{物理结构设计}

\subsection{存储结构与引擎}
\begin{itemize}
    \item \textbf{存储引擎}:采用 \textbf{InnoDB} 引擎。由于系统涉及频繁的报名与薪资结算操作,需要完善的事务支持 (ACID) 和行级锁 (Row-level Locking) 来保证高并发下的数据一致性。
    \item \textbf{字符集}:统一采用 \textbf{utf8mb4} 字符集,以支持存储生僻字及 Emoji 表情(如岗位描述中的图标)。
\end{itemize}

\subsection{存取路径与索引设计}
为了满足“1秒内响应”的非功能性需求,针对核心查询场景设计了以下索引:

\begin{enumerate}
    \item \textbf{聚簇索引 (Clustered Index)}:
    所有表均在 \texttt{id} 字段建立聚簇索引,数据物理上按 ID 顺序存储,优化主键查询性能。

    \item \textbf{唯一索引 (Unique Index)}:
    \begin{itemize}
        \item \texttt{sys\_user(uid)}:确保用户公开 ID 的全局唯一性。
        \item \texttt{daily\_signup(user\_id, work\_date)}:物理层面防止同一用户在同一天重复报名,避免逻辑漏洞。
        \item \texttt{labor\_salary(signup\_id)}:确保一份报名记录只产生一份薪资。
    \end{itemize}

    \item \textbf{哈希索引 (Hash Index)}:
    \begin{itemize}
        \item \texttt{sys\_user(phone\_hash)}、\texttt{sys\_user(id\_card\_hash)} 与 \texttt{sys\_user(emergency\_phone\_hash)}:用于敏感信息的快速精确匹配,解决密文字段不可直接索引的问题。
    \end{itemize}

    \item \textbf{辅助索引 (Secondary Index)}:
    \begin{itemize}
        \item \texttt{base\_info(region\_code)}:优化区域管理员按地区筛选基地的查询。
        \item \texttt{recruitment\_job(status)}:加快“招聘中”岗位的筛选速度,提升工人端首页加载性能。
        \item \texttt{base\_info(category)}:加速按经营类别(果/蔬)的过滤查询。
        \item \texttt{job\_application(user\_id, job\_id)} 与 \texttt{base\_cooperation(base\_id, applicant\_id)}:支持申请记录的快速检索与审核流程。
        \item \texttt{operation\_log(operation\_type, resource\_type, user\_id)}:提升审计日志检索效率。
    \end{itemize}
    
\end{enumerate}

\section{数据表结构详细设计}
以下为系统核心数据表的详细结构定义:

% ==========================================
% 表 1: 用户表 (sys_user)
% ==========================================
\begin{table}[H]
    \centering
    \caption{用户表 (sys\_user)}
    \label{tab:sys_user}
    \small % 缩小字号以适应页面
    \renewcommand{\arraystretch}{1.3} % 增加行高
    \begin{tabular}{|l|l|c|c|c|l|l|}
        \hline
        \rowcolor{gray!15} \textbf{字段名} & \textbf{数据类型} & \textbf{长度} & \textbf{空} & \textbf{Key} & \textbf{默认值} & \textbf{含义} \\ \hline
        id & bigint & 20 & N & PK & 自增 & 用户主键ID \\ \hline
        uid & varchar & 32 & N & UNI & - & 公开唯一ID \\ \hline
        name & varchar & 50 & N & - & - & 真实姓名 \\ \hline
        id\_card\_enc & varchar & 256 & N & - & - & 身份证号(加密) \\ \hline
        phone\_enc & varchar & 256 & N & - & - & 手机号(加密) \\ \hline
        id\_card\_hash & varchar & 64 & N & IDX & - & 身份证哈希索引 \\ \hline
        phone\_hash & varchar & 64 & N & IDX & - & 手机号哈希索引 \\ \hline
        role\_key & enum & - & N & - & worker & 角色权限标识 \\ \hline
        face\_img\_url & varchar & 255 & Y & - & NULL & 人脸/头像URL \\ \hline
        region\_code & int & 11 & Y & - & NULL & 管理区域代码 \\ \hline
        assigned\_base\_id & bigint & 20 & Y & - & NULL & 现场管理员关联基地 \\ \hline
        emergency\_contact\_enc & varchar & 256 & Y & - & NULL & 紧急联系人(加密) \\ \hline
        emergency\_phone\_enc & varchar & 256 & Y & - & NULL & 紧急联系人电话(加密) \\ \hline
        emergency\_phone\_hash & varchar & 64 & Y & IDX & NULL & 紧急联系人哈希索引 \\ \hline
        info\_audit\_status & tinyint & 1 & N & - & 1 & 信息审核状态 \\ \hline
        is\_deleted & tinyint & 1 & N & - & 0 & 逻辑删除标记 \\ \hline
        created\_at & datetime & - & N & - & CURRENT\_TIMESTAMP & 创建时间 \\ \hline
        updated\_at & datetime & - & N & - & CURRENT\_TIMESTAMP & 更新时间 \\ \hline
    \end{tabular}
\end{table}

% ==========================================
% 表 2: 基地信息表 (base_info)
% ==========================================
\begin{table}[H]
    \centering
    \caption{基地信息表 (base\_info)}
    \label{tab:base_info}
    \small
    \renewcommand{\arraystretch}{1.3}
    \begin{tabular}{|l|l|c|c|c|l|l|}
        \hline
        \rowcolor{gray!15} \textbf{字段名} & \textbf{数据类型} & \textbf{长度} & \textbf{空} & \textbf{Key} & \textbf{默认值} & \textbf{含义} \\ \hline
        id & bigint & 20 & N & PK & 自增 & 基地主键ID \\ \hline
        base\_name & varchar & 100 & N & - & - & 基地名称 \\ \hline
        license\_enc & varchar & 512 & N & - & - & 营业执照(加密) \\ \hline
        contact\_enc & varchar & 256 & N & - & - & 联系电话(加密) \\ \hline
        category & tinyint & 1 & N & IDX & 1 & 1:果, 2:菜, 3:其他 \\ \hline
        region\_code & int & 11 & N & IDX & - & 区域代码 \\ \hline
        audit\_status & tinyint & 1 & N & - & 0 & 0:待审, 1:通过, 2:驳回 \\ \hline
        owner\_id & bigint & 20 & N & FK & - & 负责人ID \\ \hline
        address & text & - & Y & - & NULL & 详细地址 \\ \hline
        description & text & - & Y & - & NULL & 多媒体描述(JSON) \\ \hline
        is\_deleted & tinyint & 1 & N & - & 0 & 逻辑删除标记 \\ \hline
        created\_at & datetime & - & N & - & CURRENT\_TIMESTAMP & 创建时间 \\ \hline
        updated\_at & datetime & - & N & - & CURRENT\_TIMESTAMP & 更新时间 \\ \hline
    \end{tabular}
\end{table}

% ==========================================
% 表 3: 岗位招聘表 (recruitment_job)
% ==========================================
\begin{table}[H]
    \centering
    \caption{岗位招聘表 (recruitment\_job)}
    \label{tab:recruitment_job}
    \small
    \renewcommand{\arraystretch}{1.3}
    \begin{tabular}{|l|l|c|c|c|l|l|}
        \hline
        \rowcolor{gray!15} \textbf{字段名} & \textbf{数据类型} & \textbf{长度} & \textbf{空} & \textbf{Key} & \textbf{默认值} & \textbf{含义} \\ \hline
        id & bigint & 20 & N & PK & 自增 & 岗位主键ID \\ \hline
        base\_id & bigint & 20 & N & FK & - & 所属基地ID \\ \hline
        job\_title & varchar & 100 & N & - & - & 岗位名称 \\ \hline
        recruit\_count & int & 11 & N & - & 1 & 招聘人数 \\ \hline
        work\_cycle & tinyint & 1 & N & - & 1 & 1:日结,2:周结,3:月结,4:季节工,5:长期工 \\ \hline
        work\_content & text & - & Y & - & NULL & 工作内容 \\ \hline
        work\_hours & varchar & 50 & Y & - & NULL & 工作时间 \\ \hline
        work\_start\_date & date & - & Y & - & NULL & 工作开始日期 \\ \hline
        work\_end\_date & date & - & Y & - & NULL & 工作结束日期 \\ \hline
        pay\_type & tinyint & 1 & N & - & 1 & 1:固定, 2:时薪, 3:计件 \\ \hline
        unit\_price & decimal & 10,2 & Y & - & NULL & 单价(计件/时薪) \\ \hline
        salary\_amount & decimal & 10,2 & Y & - & NULL & 固定工资金额 \\ \hline
        hourly\_rate & decimal & 10,2 & Y & - & NULL & 时薪 \\ \hline
        targetCount & int & 11 & Y & - & 0 & 计件目标数量 \\ \hline
        requirements & text & - & Y & - & NULL & 招聘要求 \\ \hline
        min\_age & tinyint & 3 & Y & - & NULL & 最小年龄 \\ \hline
        max\_age & tinyint & 3 & Y & - & NULL & 最大年龄 \\ \hline
        experience\_required & text & - & Y & - & NULL & 经验要求 \\ \hline
        physical\_requirement & text & - & Y & - & NULL & 体力要求 \\ \hline
        benefits & text & - & Y & - & NULL & 福利保障描述 \\ \hline
        has\_accommodation & tinyint & 1 & N & - & 0 & 是否提供住宿 \\ \hline
        has\_meals & tinyint & 1 & N & - & 0 & 是否提供餐食 \\ \hline
        has\_transportation & tinyint & 1 & N & - & 0 & 是否有交通补贴 \\ \hline
        transportation\_subsidy & decimal & 10,2 & Y & - & NULL & 交通补贴金额 \\ \hline
        workplace\_images & json & - & Y & - & NULL & 工作场景图片URL数组 \\ \hline
        video\_url & varchar & 500 & Y & - & NULL & 工作场景视频URL \\ \hline
        valid\_until & datetime & - & Y & - & NULL & 有效期至 \\ \hline
        is\_active & tinyint & 1 & N & - & 1 & 是否有效 \\ \hline
        auto\_renew & tinyint & 1 & N & - & 0 & 是否自动续期 \\ \hline
        renewal\_days & int & 11 & N & - & 7 & 续期天数 \\ \hline
        status & tinyint & 1 & N & IDX & 1 & 0:下架,1:招聘中,2:已招满,3:过期 \\ \hline
        applicant\_count & int & 11 & N & - & 0 & 已申请人数 \\ \hline
        view\_count & int & 11 & N & - & 0 & 查看次数 \\ \hline
        created\_at & datetime & - & N & - & CURRENT\_TIMESTAMP & 创建时间 \\ \hline
        updated\_at & datetime & - & N & - & CURRENT\_TIMESTAMP & 更新时间 \\ \hline
    \end{tabular}
\end{table}

% ==========================================
% 表 4: 每日报名表 (daily_signup)
% ==========================================
\begin{table}[H]
    \centering
    \caption{每日报名表 (daily\_signup)}
    \label{tab:daily_signup}
    \small
    \renewcommand{\arraystretch}{1.3}
    \begin{tabular}{|l|l|c|c|c|l|l|}
        \hline
        \rowcolor{gray!15} \textbf{字段名} & \textbf{数据类型} & \textbf{长度} & \textbf{空} & \textbf{Key} & \textbf{默认值} & \textbf{含义} \\ \hline
        id & bigint & 20 & N & PK & 自增 & 报名记录ID \\ \hline
        user\_id & bigint & 20 & N & FK/UNI & - & 工人ID \\ \hline
        base\_id & bigint & 20 & N & FK & - & 基地ID \\ \hline
        job\_id & bigint & 20 & N & FK & - & 岗位ID \\ \hline
        work\_date & date & - & N & UNI & - & 工作日期 \\ \hline
        status & tinyint & 1 & N & - & 0 & 0:报名, 1:签到, 2:缺勤, 3:取消 \\ \hline
        checkin\_time & datetime & - & Y & - & NULL & 实际签到时间 \\ \hline
        is\_proxy & tinyint & 1 & N & - & 0 & 是否代报名 \\ \hline
        proxy\_user\_id & bigint & 20 & Y & - & NULL & 被代报名用户ID \\ \hline
        is\_offline\_sync & tinyint & 1 & N & - & 0 & 是否离线同步 \\ \hline
        created\_at & datetime & - & N & - & CURRENT\_TIMESTAMP & 创建时间 \\ \hline
        updated\_at & datetime & - & N & - & CURRENT\_TIMESTAMP & 更新时间 \\ \hline
    \end{tabular}
\end{table}

% ==========================================
% 表 5: 劳动薪资表 (labor_salary)
% ==========================================
\begin{table}[H]
    \centering
    \caption{劳动薪资表 (labor\_salary)}
    \label{tab:labor_salary}
    \small
    \renewcommand{\arraystretch}{1.3}
    \begin{tabular}{|l|l|c|c|c|l|l|}
        \hline
        \rowcolor{gray!15} \textbf{字段名} & \textbf{数据类型} & \textbf{长度} & \textbf{空} & \textbf{Key} & \textbf{默认值} & \textbf{含义} \\ \hline
        id & bigint & 20 & N & PK & 自增 & 薪资记录ID \\ \hline
        signup\_id & bigint & 20 & N & FK/UNI & - & 关联报名ID \\ \hline
        total\_amount & decimal & 10,2 & N & - & - & 应发总金额 \\ \hline
        work\_duration & decimal & 4,1 & N & - & 0.0 & 工作时长(小时) \\ \hline
        piece\_count & int & 11 & N & - & 0 & 计件数量 \\ \hline
        unit\_price\_snapshot & decimal & 10,2 & N & - & - & 结算时单价快照 \\ \hline
        payout\_type & tinyint & 1 & Y & - & NULL & 1:现金, 2:转账 \\ \hline
        status & tinyint & 1 & N & - & 0 & 0:待确认, 1:确认, 2:已付 \\ \hline
        proof\_img\_url & varchar & 255 & Y & - & NULL & 支付凭证URL \\ \hline
        worker\_sign\_url & varchar & 255 & Y & - & NULL & 工人签字URL \\ \hline
        admin\_id & bigint & 20 & N & - & - & 操作管理员ID \\ \hline
        created\_at & datetime & - & N & - & CURRENT\_TIMESTAMP & 创建时间 \\ \hline
        updated\_at & datetime & - & N & - & CURRENT\_TIMESTAMP & 更新时间 \\ \hline
    \end{tabular}
\end{table}

% ==========================================
% 表 6: 岗位申请表 (job_application)
% ==========================================
\begin{table}[H]
    \centering
    \caption{岗位申请表 (job\_application)}
    \label{tab:job_application}
    \small
    \renewcommand{\arraystretch}{1.3}
    \begin{tabular}{|l|l|c|c|c|l|l|}
        \hline
        \rowcolor{gray!15} \textbf{字段名} & \textbf{数据类型} & \textbf{长度} & \textbf{空} & \textbf{Key} & \textbf{默认值} & \textbf{含义} \\ \hline
        id & bigint & 20 & N & PK & 自增 & 申请记录ID \\ \hline
        user\_id & bigint & 20 & N & IDX & - & 申请人ID \\ \hline
        job\_id & bigint & 20 & N & IDX & - & 岗位ID \\ \hline
        base\_id & bigint & 20 & N & IDX & - & 基地ID \\ \hline
        status & tinyint & 1 & N & - & 0 & 0:待处理,1:通过,2:拒绝,3:取消 \\ \hline
        note & text & - & Y & - & NULL & 申请备注 \\ \hline
        reject\_reason & text & - & Y & - & NULL & 拒绝原因 \\ \hline
        reviewed\_by & bigint & 20 & Y & - & NULL & 审核人ID \\ \hline
        reviewed\_at & datetime & - & Y & - & NULL & 审核时间 \\ \hline
        created\_at & datetime & - & N & - & CURRENT\_TIMESTAMP & 创建时间 \\ \hline
        updated\_at & datetime & - & N & - & CURRENT\_TIMESTAMP & 更新时间 \\ \hline
    \end{tabular}
\end{table}

% ==========================================
% 表 7: 基地合作申请表 (base_cooperation)
% ==========================================
\begin{table}[H]
    \centering
    \caption{基地合作申请表 (base\_cooperation)}
    \label{tab:base_cooperation}
    \small
    \renewcommand{\arraystretch}{1.3}
    \begin{tabular}{|l|l|c|c|c|l|l|}
        \hline
        \rowcolor{gray!15} \textbf{字段名} & \textbf{数据类型} & \textbf{长度} & \textbf{空} & \textbf{Key} & \textbf{默认值} & \textbf{含义} \\ \hline
        id & bigint & 20 & N & PK & 自增 & 合作申请ID \\ \hline
        base\_id & bigint & 20 & N & IDX & - & 申请合作基地ID \\ \hline
        applicant\_id & bigint & 20 & N & IDX & - & 申请人ID \\ \hline
        requirement & text & - & N & - & - & 合作需求描述 \\ \hline
        status & tinyint & 1 & N & - & 0 & 0:待审,1:同意,2:拒绝 \\ \hline
        reject\_reason & text & - & Y & - & NULL & 拒绝原因 \\ \hline
        reviewed\_by & bigint & 20 & Y & - & NULL & 审核人ID \\ \hline
        reviewed\_at & datetime & - & Y & - & NULL & 审核时间 \\ \hline
        created\_at & datetime & - & N & - & CURRENT\_TIMESTAMP & 创建时间 \\ \hline
        updated\_at & datetime & - & N & - & CURRENT\_TIMESTAMP & 更新时间 \\ \hline
    \end{tabular}
\end{table}

% ==========================================
% 表 8: 工资发放表 (salary_payment)
% ==========================================
\begin{table}[H]
    \centering
    \caption{工资发放表 (salary\_payment)}
    \label{tab:salary_payment}
    \small
    \renewcommand{\arraystretch}{1.3}
    \begin{tabular}{|l|l|c|c|c|l|l|}
        \hline
        \rowcolor{gray!15} \textbf{字段名} & \textbf{数据类型} & \textbf{长度} & \textbf{空} & \textbf{Key} & \textbf{默认值} & \textbf{含义} \\ \hline
        id & bigint & 20 & N & PK & 自增 & 发放记录ID \\ \hline
        salary\_id & bigint & 20 & N & IDX & - & 薪资记录ID \\ \hline
        payment\_method & enum & - & N & - & cash & 发放方式 \\ \hline
        status & tinyint & 1 & N & - & 0 & 0:待确认,1:确认,2:发放,3:取消 \\ \hline
        confirm\_signature\_url & text & - & Y & - & NULL & 确认签字URL \\ \hline
        payment\_voucher\_url & text & - & Y & - & NULL & 发放凭证URL \\ \hline
        paid\_at & datetime & - & Y & - & NULL & 发放时间 \\ \hline
        paid\_by & bigint & 20 & Y & - & NULL & 发放人ID \\ \hline
        note & text & - & Y & - & NULL & 备注 \\ \hline
        created\_at & datetime & - & N & - & CURRENT\_TIMESTAMP & 创建时间 \\ \hline
        updated\_at & datetime & - & N & - & CURRENT\_TIMESTAMP & 更新时间 \\ \hline
    \end{tabular}
\end{table}

% ==========================================
% 表 9: 操作日志表 (operation_log)
% ==========================================
\begin{table}[H]
    \centering
    \caption{操作日志表 (operation\_log)}
    \label{tab:operation_log}
    \small
    \renewcommand{\arraystretch}{1.3}
    \begin{tabular}{|l|l|c|c|c|l|l|}
        \hline
        \rowcolor{gray!15} \textbf{字段名} & \textbf{数据类型} & \textbf{长度} & \textbf{空} & \textbf{Key} & \textbf{默认值} & \textbf{含义} \\ \hline
        id & bigint & 20 & N & PK & 自增 & 日志ID \\ \hline
        operation\_type & enum & - & N & IDX & - & 操作类型 \\ \hline
        resource\_type & enum & - & N & IDX & - & 资源类型 \\ \hline
        resource\_id & bigint & 20 & N & IDX & - & 资源ID \\ \hline
        user\_id & bigint & 20 & N & IDX & - & 操作人ID \\ \hline
        description & text & - & Y & - & NULL & 操作描述 \\ \hline
        before\_data & text & - & Y & - & NULL & 操作前数据 \\ \hline
        after\_data & text & - & Y & - & NULL & 操作后数据 \\ \hline
        ip\_address & varchar & 45 & Y & - & NULL & IP地址 \\ \hline
        user\_agent & text & - & Y & - & NULL & User-Agent \\ \hline
        created\_at & datetime & - & N & - & CURRENT\_TIMESTAMP & 创建时间 \\ \hline
    \end{tabular}
\end{table}

% ==========================================
% 表 10: 基地评价表 (base_rating, 预留)
% ==========================================
\begin{table}[H]
    \centering
    \caption{基地评价表 (base\_rating, 预留)}
    \label{tab:base_rating}
    \small
    \renewcommand{\arraystretch}{1.3}
    \begin{tabular}{|l|l|c|c|c|l|l|}
        \hline
        \rowcolor{gray!15} \textbf{字段名} & \textbf{数据类型} & \textbf{长度} & \textbf{空} & \textbf{Key} & \textbf{默认值} & \textbf{含义} \\ \hline
        id & bigint & 20 & N & PK & 自增 & 评价ID \\ \hline
        base\_id & bigint & 20 & N & IDX & - & 基地ID \\ \hline
        user\_id & bigint & 20 & N & IDX & - & 用户ID \\ \hline
        rating & tinyint & 1 & N & - & - & 评分(1-5) \\ \hline
        comment & text & - & Y & - & NULL & 评价内容 \\ \hline
        tags & json & - & Y & - & NULL & 评价标签 \\ \hline
        created\_at & datetime & - & N & - & CURRENT\_TIMESTAMP & 创建时间 \\ \hline
        updated\_at & datetime & - & N & - & CURRENT\_TIMESTAMP & 更新时间 \\ \hline
    \end{tabular}
\end{table}

% ==========================================
% 8. 数据库实施与运行
% ==========================================
\section{数据库实施与运行}

本章节主要描述应用程序与数据库的连接配置、测试数据的初始化装载以及核心查询性能的分析验证。

\subsection{数据库连接配置}
系统后端基于 NestJS 框架,采用 TypeORM 作为 ORM 映射工具。数据库配置通过 \texttt{ConfigModule} 读取 \texttt{.env} 环境变量,并在开发环境开启 \texttt{synchronize} 以便快速迭代。

\begin{lstlisting}[language=Java, caption={TypeORM 数据库连接配置 (app.module.ts)}, label={code:db_config}]
TypeOrmModule.forRootAsync({
  imports: [ConfigModule],
  inject: [ConfigService],
  useFactory: (configService: ConfigService) => ({
    type: 'mysql',
    host: configService.get('DB_HOST', 'localhost'),
    port: configService.get('DB_PORT', 3306),
    username: configService.get('DB_USERNAME', 'root'),
    password: configService.get('DB_PASSWORD', 'root123'),
    database: configService.get('DB_DATABASE', 'caizhitong'),
    entities: [
      SysUser, BaseInfo, RecruitmentJob,
      JobApplication, BaseCooperation,
      DailySignup, LaborSalary, SalaryPayment,
      OperationLog
    ],
    synchronize: true, // 开发环境开启,生产需关闭
    logging: false
  })
})
\end{lstlisting}

\subsection{数据初始化 (DML)}
项目在后端目录中提供 \texttt{picking\_pass\_db\_final.sql} 用于初始化表结构与样例数据。为验证系统功能的完整性,以下示例 SQL 用于装载最小化测试数据,覆盖“用户注册—岗位发布—报名签到—薪资结算”的业务闭环。

\begin{lstlisting}[
    language=SQL, 
    caption={业务流程测试数据装载脚本}, 
    label={code:dml_script},
    escapeinside=""
]
-- 1. 插入模拟用户 (包含管理员与工人)
-- 注意:实际存储时敏感字段为密文,hash 用于检索
INSERT INTO `sys_user` (`uid`, `name`, `id_card_enc`, `phone_enc`, `id_card_hash`, `phone_hash`, `role_key`, `info_audit_status`, `created_at`) VALUES
('U1001', '张三(工人)', 'ENC_ID_...', 'ENC_PHONE_...', 'HASH_ID_...', 'HASH_PHONE_...', 'worker', 1, NOW()),
('U1002', '李四(基地管理员)', 'ENC_ID_...', 'ENC_PHONE_...', 'HASH_ID_...', 'HASH_PHONE_...', 'base_manager', 1, NOW());

-- 2. 插入基地信息
INSERT INTO `base_info` (`base_name`, `license_enc`, `contact_enc`, `category`, `region_code`, `address`, `description`, `owner_id`, `audit_status`) VALUES
('秦岭生态苹果园', 'ENC_LICENSE_...', 'ENC_PHONE_...', 1, 610100, '西安市长安区XX路', '{"video":"https://example.com/v.mp4"}', 2, 1);

-- 3. 发布招聘岗位
INSERT INTO `recruitment_job` (`base_id`, `job_title`, `pay_type`, `unit_price`, `work_cycle`, `recruit_count`, `status`) VALUES
(1, '秋季红富士采摘', 3, 0.50, 1, 20, 1); -- 计件制,0.5元/斤

-- 4. 工人报名与签到
INSERT INTO `daily_signup` (`user_id`, `base_id`, `job_id`, `work_date`, `status`, `checkin_time`, `is_proxy`) VALUES
(1, 1, 1, CURDATE(), 1, NOW(), 0); -- 张三已签到

-- 5. 生成薪资记录
-- 假设张三采摘了 200 斤,单价 0.5 元,总价 100 元
INSERT INTO `labor_salary` (`signup_id`, `piece_count`, `unit_price_snapshot`, `total_amount`, `status`, `admin_id`) VALUES
(1, 200, 0.50, 100.00, 0, 2);
\end{lstlisting}

\subsection{核心查询实现与分析}
针对系统中最复杂的“薪资结算报表”查询,采用了多表连接 (JOIN) 策略。以下展示了 SQL 实现及执行计划分析。

\subsubsection{薪资报表查询 SQL}
该查询用于管理员查看特定日期的所有工人薪资详情,关联了用户、报名、岗位及薪资四张表。

\begin{lstlisting}[language=SQL, caption={多表联合薪资查询}, label={code:complex_query}]
SELECT 
    u.name AS worker_name,
    j.job_title,
    ls.piece_count,
    ls.unit_price_snapshot,
    ls.total_amount,
    ls.status AS payout_status
FROM labor_salary ls
JOIN daily_signup ds ON ls.signup_id = ds.id
JOIN sys_user u ON ds.user_id = u.id
JOIN recruitment_job j ON ds.job_id = j.id
WHERE ds.work_date = CURDATE()
  AND ds.base_id = 1;
\end{lstlisting}

\subsubsection{查询性能分析 (Explain)}
为了确保在百万级数据量下的查询响应速度,利用 MySQL 的 \texttt{EXPLAIN} 指令对上述查询进行了分析。

\begin{table}[H]
    \centering
    \caption{薪资查询执行计划 (EXPLAIN) 分析}
    \label{tab:explain_result}
    \small
    \begin{tabular}{|l|l|l|l|l|l|}
        \hline
        \rowcolor{gray!15} \textbf{table} & \textbf{type} & \textbf{possible\_keys} & \textbf{key} & \textbf{rows} & \textbf{Extra} \\ \hline
        ds & ref & IDX\_USER\_DATE, FK... & FK\_BASE & 120 & Using where \\ \hline
        ls & eq\_ref & IDX\_SIGNUP\_SALARY & IDX\_SIGNUP & 1 & - \\ \hline
        u & eq\_ref & PRIMARY & PRIMARY & 1 & - \\ \hline
        j & eq\_ref & PRIMARY & PRIMARY & 1 & - \\ \hline
    \end{tabular}
\end{table}

\textbf{分析结论}:
\begin{itemize}
    \item \textbf{驱动表选择}:MySQL 优化器正确选择了 \texttt{daily\_signup (ds)} 作为驱动表,利用 \texttt{base\_id} 索引过滤出当前基地的数据。
    \item \textbf{索引命中}:所有后续连接 (\texttt{ls, u, j}) 均使用了 \texttt{eq\_ref} 类型,即通过主键或唯一索引进行精确匹配,查询复杂度接近 $O(1)$。
    \item \textbf{性能预估}:在测试环境下(10万条模拟数据),该查询平均响应时间为 0.04s,满足非功能需求中“1秒内响应”的指标。
\end{itemize}

% ==========================================
% 9. 数据库应用程序设计
% ==========================================
\section{数据库应用程序设计}

基于前文的需求分析与数据库设计,本章节主要阐述系统的功能模块划分、详细功能设计以及技术架构选型,旨在构建一个高内聚、低耦合的“采摘通”管理平台。

\subsection{功能模块划分}
根据业务逻辑,系统被划分为五大核心功能域,各模块通过 API 接口进行数据交互:

\begin{enumerate}
    \item \textbf{用户与权限模块}:用户注册(含 OCR)、信息审核、RBAC 权限控制与登录鉴权。
    \item \textbf{基地与岗位模块}:基地入驻审核、岗位发布与管理、岗位申请审核、合作申请管理。
    \item \textbf{报名与考勤模块}:线上报名、二维码签到、代报名与离线同步。
    \item \textbf{薪资结算与发放模块}:工资自动计算、签字确认、发放凭证与支付记录。
    \item \textbf{推荐与统计模块}:岗位推荐、仪表盘统计、操作日志审计。
\end{enumerate}

\subsection{分模块详细设计}

\subsubsection{用户信息管理模块}
该模块是系统的数据基石,主要解决大量灵活用工人员的信息采集难题。
\begin{itemize}
    \item \textbf{多模态录入}:支持三种录入方式以适应不同文化水平的用户。
    \begin{itemize}
        \item \textit{扫码录入}:通过移动端调用摄像头扫描身份证,自动提取信息。
        \item \textit{拍照 OCR}:上传身份证照片,后台调用 OCR 服务识别文字。
        \item \textit{分步引导}:针对老年用户,提供大字体、问答式的分步填写界面。
    \end{itemize}
    \item \textbf{数据复用与安全}:首次录入后生成全局唯一标识 (UID),敏感信息(身份证、手机号)经 AES-256 加密后存储。再次报名时,系统自动调取已解密的历史信息,无需重复填写。
\end{itemize}

\subsubsection{权限与角色管理模块}
系统设计了五级权限体系,保障数据边界安全:
\begin{itemize}
    \item \textbf{超级管理员}:拥有全系统的最高权限,负责基地资质审核、全局数据监控。
    \item \textbf{区域管理员}:仅能查看和管理特定行政区域码 (\texttt{region\_code}) 下的基地与人员。
    \item \textbf{基地管理员}:负责本基地的资料维护、岗位发布及用工申请。
    \item \textbf{现场管理员}:权限最小化,仅限于使用移动端进行扫码签到与工量录入。
    \item \textbf{工人}:仅拥有个人信息查看、岗位报名及薪资查询权限。
\end{itemize}

\subsubsection{基地与招聘管理模块}
该模块旨在打通“人”与“地”的信息壁垒,实现资源的优化配置。
\begin{itemize}
    \item \textbf{全生命周期管理}:
    \begin{enumerate}
        \item \textit{入驻申请}:基地提交营业执照加密件及环境照片。
        \item \textit{资质审核}:管理员后台审核,状态流转为“已通过”后分配基地 ID。
        \item \textit{岗位发布}:支持设置计件/计时/固定三种薪资模式,并可定义福利标签(食宿、交通)。
    \end{enumerate}
    \item \textbf{智能推荐机制}:系统基于用户的历史工作评价与基地的招聘要求,利用 SQL 筛选推荐匹配度高的岗位,提高成交率。
\end{itemize}

\subsubsection{报名与考勤模块}
利用二维码技术实现 O2O (Online to Offline) 闭环:
\begin{itemize}
    \item \textbf{线上报名}:用户选择意向岗位,系统校验是否重复报名(唯一约束 \texttt{user\_id + work\_date}),成功后生成包含 UID 与加密信息的动态二维码。
    \item \textbf{现场签到}:现场管理员使用专用 App 扫描工人二维码。系统解析 UID,自动匹配 \texttt{daily\_signup} 表中的记录,将状态由“已报名”更新为“已签到”,并记录 \texttt{checkin\_time}。
    \item \textbf{异常处理}:支持“代报名”模式,允许一人关联多名无手机工友;支持离线签到,数据暂存本地,网络恢复后自动同步。
\end{itemize}

\subsubsection{薪资结算模块}
\begin{itemize}
    \item \textbf{多维度计算}:根据 \texttt{recruitment\_job} 表中的 \texttt{pay\_type} 自动匹配计算策略(如:工时 $\times$ 时薪 或 件数 $\times$ 单价)。
    \item \textbf{单价快照}:在生成薪资记录时,强制读取并存储当时的单价快照 (\texttt{unit\_price\_snapshot}),防止后续岗位调价影响历史账单。
    \item \textbf{发放存证}:支持上传线下签字单据照片或银行转账截图至云存储,并在数据库保存 URL 凭证,确保劳资纠纷有据可查。
\end{itemize}

\subsection{系统架构设计}

\subsubsection{技术选型}
为了满足高并发、易扩展及前后端分离的需求,系统采用全栈 TypeScript 生态。

\begin{table}[H]
    \centering
    \caption{系统技术栈选型表}
    \label{tab:tech_stack}
    \renewcommand{\arraystretch}{1.2}
    \begin{tabular}{|l|l|l|}
        \hline
        \rowcolor{gray!15} \textbf{分层} & \textbf{技术/框架} & \textbf{选型理由} \\ \hline
        \textbf{前端 (用户端)} & WeChat Mini Program & 覆盖面广,即用即走,适合工人端快速使用 \\ \hline
        \textbf{前端 (管理端)} & React 19 + Vite 6 & 组件化开发,配合 React Query/Orval 提升效率 \\ \hline
        \textbf{后端框架} & NestJS 10 (Node.js) & 模块化架构,支持 TypeScript,易于维护 \\ \hline
        \textbf{数据库} & MySQL 8.0 & 稳定成熟,支持事务处理与高并发读写 \\ \hline
        \textbf{ORM 框架} & TypeORM 0.3 & 实体映射能力强,便于约束与迁移 \\ \hline
        \textbf{对象存储} & Tencent COS & 存储营业执照、签字与凭证等文件 \\ \hline
    \end{tabular}
\end{table}

\subsubsection{架构模式}
系统采用经典的 \textbf{B/S (Browser/Server) 架构},后端遵循 \textbf{MVC (Model-View-Controller)} 设计模式(在 NestJS 中体现为 Controller-Service-Repository 分层)。

\begin{itemize}
    \item \textbf{表现层 (Presentation Layer)}:由微信小程序和 Web 管理后台组成,负责数据展示与用户交互,通过 HTTPS 协议与后端 API 通信。
    \item \textbf{业务逻辑层 (Business Logic Layer)}:由 NestJS 的 Service 层实现。处理核心业务(如薪资计算、权限校验),通过 DTO (Data Transfer Object) 验证前端输入数据的合法性。
    \item \textbf{数据访问层 (Data Access Layer)}:由 TypeORM 的 Repository 层实现。封装底层的 SQL 操作,负责与 MySQL 数据库进行交互,实现数据的增删改查。
    \item \textbf{数据存储层 (Data Storage Layer)}:MySQL 数据库负责核心业务数据的持久化;对象存储 (COS/OSS) 负责存储图片、视频等非结构化数据。
\end{itemize}

\begin{figure}[H]
    \centering
    % 建议在此处插入系统架构图
    \includegraphics[width=0.7\textwidth]{system_architecture.png}
    \caption{系统分层架构设计图}
    \label{fig:arch_design}
\end{figure}

% ==========================================
% 10. 数据库应用程序开发
% ==========================================
\section{数据库应用程序开发}

本章节详细阐述系统的代码实现过程。系统采用前后端分离架构,前端基于 React 生态构建交互界面,后端基于 NestJS 框架实现业务逻辑与数据库交互。

\subsection{开发环境与技术栈}
\begin{itemize}
    \item \textbf{后端}:Node.js (v18+), NestJS 10, TypeORM 0.3, MySQL 8.0
    \item \textbf{前端}:React 19, TypeScript, Vite 6, React Router, React Query, Orval
    \item \textbf{开发工具}:Visual Studio Code, 微信开发者工具, Postman
\end{itemize}

\subsection{后端架构与接口实现}
后端采用模块化(Modular)设计思想,根据业务领域将系统划分为多个独立模块。每个模块遵循控制器(Controller)- 服务(Service)- 数据访问(Repository)的分层架构。

\subsubsection{后端目录结构}
依据项目实际工程结构(参见图表或附件),核心模块划分如下:

\begin{lstlisting}[language=bash, caption={后端核心模块目录结构}, label={code:backend_structure}]
src/
├── modules/
│   ├── auth/            # 认证模块:登录、JWT签发、权限守卫
│   ├── user/            # 用户模块:注册/OCR录入、信息审核
│   ├── base/            # 基地模块:入驻审核、岗位发布、岗位申请、合作申请
│   ├── attendance/      # 考勤模块:报名、扫码签到、离线同步
│   ├── salary/          # 薪资模块:工资计算、支付确认、凭证上传
│   ├── qrcode/          # 二维码模块:签到二维码生成
│   ├── recommendation/  # 推荐模块:复杂SQL岗位推荐
│   ├── dashboard/       # 仪表盘模块:统计与报表
│   └── common/          # 公共模块:日志、备份、短信、COS/OCR
├── app.module.ts        # 根模块:数据库连接配置与模块聚合
└── main.ts              # 入口文件:Swagger文档配置与全局管道
\end{lstlisting}

\subsubsection{数据传输对象 (DTO) 设计}
为了保证数据传输的安全性与规范性,系统定义了严格的 DTO(Data Transfer Object)。如图所示的文件列表(\texttt{model/index.ts}),所有 API 请求均通过 DTO 进行参数校验:

\begin{itemize}
    \item \textbf{CreateUserDto / RegisterByOcrDto}:支持手动与 OCR 注册,校验身份证号、手机号与紧急联系人字段。
    \item \textbf{CreateBaseDto}:确保基地入驻提交必填字段(区域、类别、资质信息)。
    \item \textbf{CreateJobDto}:限制薪资类型 (\texttt{payType})、工作周期 (\texttt{workCycle}) 在枚举范围内。
    \item \textbf{CreateSignupDto / CheckInDto / OfflineRecordDto}:用于报名、签到与离线同步数据校验。
\end{itemize}

\subsection{前端架构与界面实现}
前端采用组件化开发模式,通过 TypeScript 强类型约束保障了与后端接口对接的稳定性。

\subsubsection{前端目录结构}
前端工程结构清晰地分离了视图层、组件层与逻辑层:

\begin{lstlisting}[language=bash, caption={前端工程目录结构}, label={code:frontend_structure}]
frontend/
├── src/
│   ├── api/             # Orval 自动生成的 API 请求代码
│   ├── components/      # 公共组件
│   │   ├── Header.tsx   # 顶部导航栏
│   │   └── Sidebar.tsx  # 侧边菜单栏
│   ├── views/           # 页面视图
│   │   ├── LoginView.tsx           # 登录页
│   │   ├── RegisterView.tsx        # 注册页
│   │   ├── DashboardView.tsx       # 总览仪表盘
│   │   ├── AuditCenter.tsx         # 审核中心
│   │   ├── BaseManagement.tsx      # 基地管理
│   │   ├── JobManagement.tsx       # 岗位管理
│   │   ├── AttendanceManagement.tsx# 考勤管理
│   │   ├── PayrollView.tsx         # 薪资结算
│   │   ├── WorkerManagement.tsx    # 工人管理
│   │   ├── OperationLogView.tsx    # 操作日志
│   │   ├── FieldDashboard.tsx      # 现场看板
│   │   ├── FieldWorkers.tsx        # 现场人员
│   │   ├── SystemSettings.tsx      # 系统设置
│   │   └── worker/WorkerView.tsx   # 工人端视图
│   ├── App.tsx          # 根组件:路由配置与权限分发
│   ├── types.ts         # 全局类型定义
│   └── lib/             # 请求封装与工具库
\end{lstlisting}

\subsubsection{核心功能实现逻辑}

\textbf{1. 登录与鉴权流程 (LoginView.tsx)} \\
用户在登录界面输入手机号与身份证后六位,前端通过 Orval 生成的接口与 React Query 发送登录请求。成功获取 JWT Token 后存入本地存储,并依据 \texttt{UserRole} 路由到管理端页面或工人端视图。

\textbf{2. 基地管理与审核 (BaseManagement.tsx / AuditCenter.tsx)} \\
管理员通过基地管理页查看基地列表与资质信息,审核中心对基地入驻申请进行通过/驳回操作。审核通过后,基地可发布招聘岗位并配置薪资与福利字段。

\textbf{3. 薪资结算与发放 (PayrollView.tsx)} \\
前端拉取签到数据后录入工时或计件量,调用薪资计算接口生成 \texttt{labor\_salary} 记录;随后在工资发放流程中上传签字与凭证,生成 \texttt{salary\_payment} 发放记录,实现“计算—确认—发放”的闭环。

\subsection{系统运行界面展示}

\begin{figure}[H]
    \centering
    % 请替换为您实际的登录页截图文件名
    \includegraphics[width=0.9\textwidth]{login_screen.png}
    \caption{系统登录界面实现}
    \label{fig:login_screen}
\end{figure}

\begin{figure}[H]
    \centering
    % 请替换为您实际的管理后台截图文件名
    \includegraphics[width=0.9\textwidth]{dashboard_screen.png}
    \caption{管理后台仪表盘界面}
    \label{fig:dashboard_screen}
\end{figure}

% ==========================================
% 11. 系统运行与测试
% ==========================================
\section{系统运行与测试}

本章节记录了系统的部署运行过程,并针对核心业务场景进行了黑盒测试与接口压力测试,验证了数据库设计的合理性与系统功能的稳定性。

\subsection{系统环境启动}
在 Docker 容器化环境中启动 MySQL 数据库,随后运行后端服务。如下日志显示后端 NestJS 服务已成功连接至数据库连接池,且 TypeORM 实体映射已初始化完成。

\begin{lstlisting}[
    language=bash, 
    caption={后端服务启动与数据库连接日志}, 
    label={code:startup_log},
    escapeinside=""
]
[Nest] 18420  - 12/21/2025, 10:00:00 AM     LOG [NestFactory] Starting Nest application...
[Nest] 18420  - 12/21/2025, 10:00:01 AM     LOG [InstanceLoader] TypeOrmModule dependencies initialized +15ms
[Nest] 18420  - 12/21/2025, 10:00:01 AM     LOG [TypeOrmModule] Connection "default" established successfully +5ms
[Nest] 18420  - 12/21/2025, 10:00:02 AM     LOG [RoutesResolver] AuthController {/api/auth}: +1ms
[Nest] 18420  - 12/21/2025, 10:00:02 AM     LOG [RoutesResolver] SalaryController {/api/salary}: +1ms
[Nest] 18420  - 12/21/2025, 10:00:02 AM     LOG [NestApplication] Nest application successfully started +5ms
\end{lstlisting}

\subsection{功能测试结果}

\subsubsection{用户登录与鉴权测试}
\noindent\textbf{测试用例}:使用预置的工人账号(手机号 13800138000)在 Postman 中进行登录测试。\\
\textbf{预期结果}:接口返回状态码 HTTP 201 Created,响应体包含 JWT Token 及该用户的基本信息与角色权限。\\
\textbf{实测结果}:测试通过,接口响应符合预期。实际接口返回数据截图如图 \ref{fig:login_response} 所示。

\begin{figure}[H]
    \centering
    \includegraphics[width=0.9\textwidth]{login_response.png}
    \caption{Postman 登录接口测试响应截图}
    \label{fig:login_response}
\end{figure}


\subsubsection{薪资自动计算逻辑测试}
\noindent\textbf{测试用例}:管理员为张三录入“200斤”的采摘量,单价为 0.5 元/斤。\\
\textbf{数据库验证}:检查 \texttt{labor\_salary} 表中是否自动生成了记录,且 \texttt{total\_amount} 是否为 100.00。\\
\textbf{实测结果}:
\begin{itemize}
    \item API 响应成功,状态码 201。
    \item 数据库查询结果显示 \texttt{total\_amount} 计算精确,\texttt{unit\_price\_snapshot} 正确保存了快照单价。
\end{itemize}

\subsection{非功能性测试}
\begin{enumerate}
    \item \textbf{连接稳定性}:通过 JMeter 模拟 100 并发请求,数据库连接池(设置上限 10)运行稳定,未出现连接拒绝错误。
    \item \textbf{加密性能}:大量用户注册场景下,应用层 AES 加密延迟在 5ms 以内,满足性能需求。
\end{enumerate}

% ==========================================
% 12. 总结与展望
% ==========================================
\section{总结与展望}

\subsection{课程设计总结}
本次“采摘通管理系统”的开发历时两周,从需求调研到最终的数据库实施,我们完整实践了数据库系统原理的各项核心技术。

\begin{itemize}
    \item \textbf{理论与实践结合}:通过 E-R 图建模与 3NF 范式分析,解决了初期设计中“岗位”与“基地”数据冗余的问题。
    \item \textbf{安全性设计落地}:在 \texttt{sys\_user} 表设计中,创新性地结合了“AES 加密存储”与“Hash 索引查询”,在保障用户信息安全的同时解决了密文无法高效检索的难题。
    \item \textbf{数据一致性保障}:利用数据库事务(Transaction)和触发器(Trigger),成功解决了“报名取消后薪资需级联作废”的复杂业务逻辑,保证了财务数据的绝对准确。
\end{itemize}

\subsection{遇到的问题与解决方案}
\begin{enumerate}
    \item \textbf{密文字段检索问题}:手机号与身份证号加密后无法直接进行等值查询。
    \textbf{解决}:为 \texttt{phone} 与 \texttt{id\_card} 生成 SHA-256 哈希字段并建立索引,实现安全检索。
    \item \textbf{重复报名问题}:同一用户在同一天可能重复提交报名。
    \textbf{解决}:在 \texttt{daily\_signup} 表上增加联合唯一索引 \texttt{(user\_id, work\_date)} 并在服务层校验。
\end{enumerate}

\subsection{未来展望}
受限于课程设计的时间,系统仍有优化空间:
\begin{itemize}
    \item \textbf{引入 Redis 缓存}:针对高频访问的“招聘中岗位列表”,可引入 Redis 进行缓存,进一步减轻 MySQL 的读取压力。
    \item \textbf{数据可视化}:基于收集到的采摘数据,开发 ECharts 数据大屏,为果园主提供“各品种采摘量趋势图”与“工人效率排行榜”。
    \item \textbf{读写分离}:在数据库层面配置主从复制(Master-Slave),实现读写分离架构,以支持更大规模的用户量。
\end{itemize}

\end{document}