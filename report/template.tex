\documentclass[UTF8,a4paper]{ctexart}

\usepackage{geometry}
\usepackage{setspace}
\usepackage{fancyhdr}
\usepackage{titlesec}
\usepackage{graphicx}
\usepackage{booktabs}

% Page layout
\geometry{left=2.5cm,right=2.5cm,top=2cm,bottom=2cm}
\setlength{\parindent}{2em}
\setstretch{1.5}

% Fonts
\setCJKmainfont{SimSun}
\ctexset{
  section = {format=\heiti\zihao{2}},
  subsection = {format=\heiti\zihao{3}},
  subsubsection = {format=\heiti\zihao{4}}
}

% Footer page number (centered, small-five)
\pagestyle{fancy}
\fancyhf{}
\fancyfoot[C]{\songti\zihao{5}\thepage}

\begin{document}
\songti\zihao{-4}

\begin{titlepage}
\thispagestyle{empty}

\begin{center}
  \vspace*{2cm}
  {\heiti\zihao{3} 西北工业大学《移动应用开发》}\\[1.5em]
  {\heiti\zihao{2} 大作业报告}
\end{center}

\vspace{5cm}

\noindent
题目:

\vspace{2cm}

\noindent
\begin{tabular}{@{}ll}
学生姓名:& 赵张阳\\
学号:& 2023302782\\
班级:& 10012301班\\
指导教师:& 周果清\\
提交日期:& 2026年02月07日
\end{tabular}
\end{titlepage}

\section{选题依据}
本报告为《移动应用开发》课程大作业,重点聚焦微信小程序端的设计与实现,题目为“采摘工智慧管理系统(小程序端)”。该选题来源于真实的农业用工场景:基地招聘信息分散、工人报名流程繁琐、现场签到容易出错、工资信息反馈不及时。小程序具有“易触达、低安装成本、跨平台”的天然优势,适合用于农忙用工场景的移动端协作。通过小程序实现报名、扫码签到、考勤与工资查询,可显著提升现场管理效率与数据可追溯性,具有实际应用价值。

本项目与课程关联性突出:小程序端包含典型的移动应用交互流程(登录与注册、列表与详情、扫码能力、表单校验、角色化入口与自定义 TabBar),同时涉及网络请求、异步处理、页面生命周期、状态管理与用户体验优化。项目既体现了移动端开发的核心技术点,也体现了“多角色、多场景、强交互”的应用特征,符合移动应用开发课程的培养目标。

\section{设计思路}
\subsection{开发目标}
系统采用“小程序端 + 网页管理端”协同模式:小程序端提供稳定、低门槛的移动入口,覆盖采摘工与现场管理员的高频操作,形成“报名—签到—考勤—薪资确认”的闭环流程;网页管理端负责基地审核、岗位管理、考勤与薪资统计、系统配置等管理类任务。面向采摘工侧,强调岗位信息可见、报名流程简化与个人信息可追溯;面向现场管理员侧,强调扫码签到效率与考勤记录可查询;面向管理端侧,强调数据集中、流程可控与统计可视化。整体目标是通过移动端提升现场效率、通过网页端提升管理效率,并保证数据一致性与可追溯性。

\subsection{核心功能}
核心功能分为小程序端与网页端两部分:小程序端包括采摘工端(登录注册、基地与岗位浏览、岗位报名、签到二维码展示、个人资料维护、考勤记录与薪资查询)与现场管理员端(工作台概览、扫码签到、考勤记录查询与个人信息);网页管理端包括基地入驻审核、岗位发布与管理、报名审核、考勤统计、薪资管理、用户与权限管理、数据看板。功能设计遵循“少步骤、强反馈、易操作”的原则,确保现场场景下可快速完成任务,同时保证管理端流程清晰、数据可视化。

\subsection{整体架构}
整体架构由网页管理端、小程序端与后端服务组成。小程序采用原生框架,页面层负责展示与交互,服务层通过统一的 \texttt{request} 方法访问后端 API,数据层依赖本地缓存与页面状态管理;网页端采用组件化页面结构,通过统一 API 层获取数据并实现审核与统计功能。架构上小程序按角色划分入口,通过自定义 TabBar 进行角色化导航;页面间使用 \texttt{wx.navigateTo} 与 \texttt{wx.switchTab} 形成清晰的操作路径;关键页面通过 \texttt{onShow} 触发数据刷新,保证状态一致。

\subsection{技术选型}
小程序端采用微信原生小程序技术栈,理由是兼容性好、部署成本低、与微信生态能力(扫码、缓存、登录态)结合紧密;网页端采用 React + Vite + TypeScript 以提升管理端的可维护性与开发效率;网络请求统一封装,便于添加鉴权、错误处理与重试逻辑;UI 采用卡片化布局与统一色彩体系,提升可读性与一致性。为降低运行时问题,小程序端避免使用依赖 Babel 运行时的语法特性,保证多机型稳定运行。

\subsection{模块划分}
模块划分以功能与角色为主线:小程序端包含认证模块(登录/注册/资料完善)、岗位与基地模块(列表与详情)、报名模块(申请与状态)、考勤模块(二维码与扫码签到)、个人中心模块(资料、记录与薪资)、现场管理员模块(工作台、扫码、记录);网页端包含基地管理、岗位管理、报名审核、考勤统计、薪资管理、用户与权限管理、数据看板等模块。同时提供公共模块:网络请求封装、权限与角色缓存、错误提示与加载状态。

\subsection{流程设计}
流程一(采摘工端):登录后进入首页,浏览岗位与基地,提交报名,进入个人中心查看报名状态与签到二维码。\\
流程二(现场管理员端):通过扫码签到获取工人信息,生成考勤记录,并可按日期查询考勤明细。\\
流程三(网页管理端):基地管理员审核入驻与岗位申请,管理岗位发布与报名审核;管理端查看考勤与薪资统计,形成报表与决策支持。\\
流程四(闭环确认):采摘工在个人中心查看考勤与薪资信息,完成确认流程,形成可追溯链路。\\
流程设计强调“任务可闭环、状态可追溯、反馈可见”,降低现场操作成本,同时保证管理端可控与可审计。

\subsection{创新性与逻辑性}
创新性体现在角色化导航与多端闭环:通过自定义 TabBar 将采摘工与现场管理员分流,确保入口清晰;通过扫码与考勤统计形成现场闭环数据链;通过网页端审核与统计形成管理闭环。逻辑性体现在“角色—功能—流程”的层层递进:先定义角色职责,再确定核心功能,再通过页面与接口串联流程,保证实现路径清晰可落地。

\section{功能需求与实现}
\subsection{采摘工端功能}
采摘工端主要功能包括:登录与注册、基地浏览、岗位查看与报名、签到二维码展示、个人资料维护、考勤记录查询、工资统计与确认。登录成功后缓存用户信息与角色,首页展示推荐基地与岗位,报名页展示申请状态,个人中心提供资料完善入口和工资确认入口。

\subsection{现场管理员端功能}
现场管理员端功能包括:工作台(基地信息与考勤概况)、扫码签到、考勤记录、个人信息。扫码签到支持 \texttt{wx.scanCode} 与手动输入,签到结果写入历史列表,便于现场快速核对。

\subsection{网页管理端功能}
网页管理端功能包括:基地入驻审核、岗位发布与管理、报名审核、考勤统计、薪资管理、用户与权限管理、数据看板与操作日志。通过管理端可完成业务审核与统计汇总,保证流程可控、数据可追溯,并为运营决策提供数据支持。

\subsection{角色化入口与导航}
自定义 TabBar 根据角色动态渲染不同导航项。登录后根据角色跳转到不同首页,进入页面时通过生命周期函数刷新 TabBar 角色状态,避免角色切换时显示错误菜单。

\section{小程序端实现过程}
\subsection{环境搭建}
配置微信开发者工具,导入小程序项目;设置小程序基础配置(\texttt{app.json}、\texttt{project.config.json});配置网络请求基地址与必要权限;建立统一请求方法处理 token、错误码与异常。

\subsection{核心页面开发}
按“登录/注册—首页—岗位与报名—签到码—个人中心—现场管理员功能”顺序实现。每个页面包括:数据初始化、接口请求、交互事件绑定与 UI 展示;关键页面使用 \texttt{onShow} 触发数据刷新,保证页面状态及时更新。

\subsection{关键难点与解决}
难点一是角色化 TabBar 的动态刷新,采用组件 \texttt{pageLifetimes.show} 重新读取角色信息;难点二是扫码签到的异常处理,增加失败提示与历史记录;难点三是小程序运行时兼容性问题,避免使用需要 Babel 运行时的语法特性,降低运行错误风险。

\section{接口对接与数据流}
小程序端通过统一请求方法对接后端接口,关键接口包括:登录与注册、基地与岗位查询、报名申请、签到二维码获取、签到提交、考勤记录查询、薪资统计与确认。请求过程统一处理 token、超时与错误提示,确保用户体验一致。

\section{测试与优化}
\subsection{功能测试}
覆盖登录注册、岗位浏览与报名、二维码生成、扫码签到、考勤查询、工资确认等核心流程。测试场景包括正常流程、网络异常、非法输入与权限不足。

\subsection{体验优化}
针对移动端交互特点优化表单校验提示、按钮状态与加载态反馈;对列表页增加空状态提示;对扫码流程增加结果卡片与历史记录,提升现场使用效率。

\section{总结与展望}
本课程大作业以小程序端为核心,完成了采摘工与现场管理员两类角色的移动端应用开发,具备完整的报名、签到、查询与确认链路。项目体现了小程序开发中的关键技术点与实际业务场景结合。后续可扩展方向包括:加入地图定位与导航、离线签到补传、消息订阅与通知推送、薪资电子签名与支付对接等。

\section{对课程的建议}
建议课程增加小程序实战环节比例,强化从需求分析到页面实现的完整训练;增加扫码、定位、授权等移动端能力的专项练习;提供更多真实业务场景的案例与测试模板,帮助同学提升实战能力。

\section{附录}
1. 核心代码片段(关键功能实现代码,标注说明):\\
\quad 小程序端:\texttt{miniprogram/pages/}、\texttt{miniprogram/custom-tab-bar/}。\\
2. 小程序运行截图/演示视频链接(含运行环境、核心功能演示画面):可补充登录、首页、岗位详情、扫码签到、工资确认等关键画面。\\
3. 参考文献(开发过程中参考的官方文档、技术博客等):\\
\quad 微信小程序开发文档、微信开放能力文档、前端工程化与用户体验设计资料。

\end{document}
